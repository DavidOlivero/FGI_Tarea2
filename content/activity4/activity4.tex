\newpage
\clearpage
\justify
\fontsize{11pt}{14}\
\setlength{\parindent}{0cm}
\hbadness=10000

\section{Actividad 4}
\begin{longtable}{|m{4cm}|m{5.3cm}|m{4.2cm}|}
    \hline
    \multicolumn{1}{|c|}{\textbf{Principio}} & \multicolumn{1}{|c|}{\textbf{Definición}} & \multicolumn{1}{|c|}{ \textbf{Ejemplo}} \\ [5pt]
    \hline
    \centering \textbf{Los individuos enfrentan trade-offs} & Las personas y la sociedad en su conjunto enfrentan la necesidad de tomar decisiones donde, al elegir algo, están renunciando a otra opción. & La decisión de una sociedad de asignar más recursos a la educación podría implicar sacrificar la inversión en infraestructura. \\
    \hline
    \centering \textbf{El coste de oportunidad} & Al tomar una decisión, el coste de oportunidad es el valor de lo que se renuncia al elegir una alternativa sobre otra. & Si decides pasar tu tiempo estudiando en lugar de trabajar, el coste de oportunidad sería el salario perdido. \\ [5pt]
    \hline
    \centering \textbf{Los individuos racionales piensan en márgenes} & Las personas toman decisiones evaluando los beneficios y costos marginales. & El dueño de una tienda de camisetas decide producir una camiseta más, para ello buscará que los costes de producción adicionales sean menores a las ganancias adicionales. \\ [5pt]
    \hline
    \centering \textbf{Los individuos responden a incentivos} & Los incentivos, tanto positivos como negativos, influyen en el comportamiento humano. & Los impuestos más bajos pueden incentivar a las personas a trabajar más, mientras que los impuestos más altos pueden desincentivar el trabajo adicional. \\ [5pt]
    \hline
    \centering \textbf{El comercio puede mejorar el bienestar de todos} & El comercio permite a las personas especializarse en lo que hacen mejor y luego intercambiar esos bienes y servicios por lo que necesitan o desean. & Una persona es habil fabricando muebles de madera, por lo que si decide comerciar con esto, podrá hacer un mueblo a a menos costo que otra persona con menos habilidad. \\ [5pt]
    \hline
    \centering \textbf{Los mercados son en general una buena manera de organizar la actividad económica} & En muchos casos, los mercados son eficientes para asignar recursos. La competencia y los precios actúan como señales para coordinar las decisiones de los productores y consumidores. & Si hay variedad de productos y competitividad, indicadores económicos como la oferta y la demanda se regulan. \\ [5pt]
    \hline
    \centering \textbf{El Estado puede mejorar a veces los resultados del mercado} & Las decisiones y regulaciones que haga el estado puede beneficiar al comercio, tales como regulaciones de impuestos u otros factores. & En casos de externalidades (como la contaminación) o bienes públicos (como la defensa nacional), el gobierno puede intervenir para corregir esas fallas. \\ [5pt]
    \hline
    \centering \textbf{El nivel de vida de un país depende de su capacidad para producir bienes y servicios} & La productividad, medida como la cantidad de bienes y servicios producidos por cada unidad de trabajo, es crucial para determinar el nivel de vida de una sociedad a largo plazo. & En el país A, la producción por trabajador es alta debido a la disponibilidad de tecnología avanzada, una fuerza laboral educada y bien capacitada, y una infraestructura moderna. Como resultado, el país A puede producir una amplia gama de bienes y servicios a precios competitivos en el mercado global. \\ [5pt]
    \hline
    \centering \textbf{Los precios suben cuando el gobierno imprime demasiado dinero} & La inflación, un aumento generalizado de los precios, es causada en parte por el crecimiento excesivo de la oferta monetaria. Cuando hay demasiado dinero en circulación, su valor disminuye, lo que lleva a un aumento de los precios. & Una economía donde el gobierno decide imprimir grandes cantidades de dinero para financiar sus gastos, como programas de infraestructura, salud y educación, sin un respaldo adecuado en la producción de bienes y servicios. \\ [5pt]
    \hline
    \centering \textbf{La sociedad enfrenta una disyuntiva entre la inflación y el desempleo a corto plazo} & La política económica puede influir en la tasa de inflación y la tasa de desempleo, pero a menudo hay un trade-off entre ellas. & Una política monetaria expansiva puede reducir el desempleo a corto plazo para aumentar la inflación. Si señor \\ [5pt]
    \hline
\end{longtable}