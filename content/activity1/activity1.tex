
\justify
\fontsize{12pt}{14}\
\setlength{\parindent}{0cm}

\normalsize En el contexto de la vida diaria siempre se maneja la economía como una manera de regualar o administrar los recursos de una familia, persona, negocio, etc. Así lo deja calaro lo expresado por \cite{pereira2011} que indica que la actividad humana se realiza buscando el máximo rendimiento, es decir, la utilización de los mínimos recursos, en condiciones de ética y moral. Por tal motivo, a través de la actividad económica se busca proveer del mayor bienestar a los consumidores de bienes y servicios, y satisfacer las necesidades humanas en un mundo donde los recursos son cada vez más escasos.

\normalsize En este sentido, según lo dicho por \cite{llorca2016} la economía puede ser definida como la disciplina que analiza la gestión de recursos limitados con el fin de producir una variedad de bienes y servicios, los cuales son distribuidos para su consumo dentro de la sociedad. Así pues, se puede entender como un bien a cualquier cosa que directa o indirectamente satisfaga las necesidades o deseos de las personas.

\normalsize De este modo, tal como lo referencia \cite{euroinnova2023} Para cualquier individuo, comprender los principios económicos y su influencia en la vida diaria resulta esencial. Lo anterior, se debe a que, esta disciplina ejerce una influencia directa en las decisiones cotidianas, por lo que familiarizarse con sus conceptos es crucial. De tal forma que, al obtener un conocimiento económico sólido, las personas pueden tomar decisiones financieras más acertadas, analizar críticamente las políticas gubernamentales que impactan en su bienestar y entender el funcionamiento de los mercados y las empresas en su entorno.

\normalsize Además, entender la economía permite participar activamente en debates públicos sobre cuestiones económicas y contribuir al desarrollo de una sociedad más próspera y equitativa. Por ende, la economía actúa como una herramienta esencial para comprender cómo se crean, distribuyen y utilizan los recursos en la sociedad, lo que facilita la adopción de decisiones más informadas y responsables, tanto para el beneficio individual como para el bienestar común \parencite{euroinnova2023}.
