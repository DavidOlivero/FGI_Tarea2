
\justify
\fontsize{12pt}{14}\
\setlength{\parindent}{0cm}

\newpage
\section{Actividad 1}
\normalsize En el contexto de la vida diaria siempre se maneja la economía como una manera de regualar o administrar los recursos de una familia, persona, negocio, etc. Así lo deja calaro lo expresado por \cite{pereira2011} que indica que la actividad humana se realiza buscando el máximo rendimiento, es decir, la utilización de los mínimos recursos, en condiciones de ética y moral. Por tal motivo, a través de la actividad económica se busca proveer del mayor bienestar a los consumidores de bienes y servicios, y satisfacer las necesidades humanas en un mundo donde los recursos son cada vez más escasos.

\normalsize En este sentido, según lo dicho por \cite{llorca2016} la economía puede ser definida como la disciplina que analiza la gestión de recursos limitados con el fin de producir una variedad de bienes y servicios, los cuales son distribuidos para su consumo dentro de la sociedad. Así pues, se puede entender como un bien a cualquier cosa que directa o indirectamente satisfaga las necesidades o deseos de las personas.

\normalsize De este modo, tal como lo referencia \cite{euroinnova2023} Para cualquier individuo, comprender los principios económicos y su influencia en la vida diaria resulta esencial. Lo anterior, se debe a que, esta disciplina ejerce una influencia directa en las decisiones cotidianas, por lo que familiarizarse con sus conceptos es crucial. De tal forma que, al obtener un conocimiento económico sólido, las personas pueden tomar decisiones financieras más acertadas, analizar críticamente las políticas gubernamentales que impactan en su bienestar y entender el funcionamiento de los mercados y las empresas en su entorno.

\normalsize Además, entender la economía permite participar activamente en debates públicos sobre cuestiones económicas y contribuir al desarrollo de una sociedad más próspera y equitativa. Por ende, la economía actúa como una herramienta esencial para comprender cómo se crean, distribuyen y utilizan los recursos en la sociedad, lo que facilita la adopción de decisiones más informadas y responsables, tanto para el beneficio individual como para el bienestar común \parencite{euroinnova2023}.

% Original tex \normalsize Ahora bien, teniendo todo esto en cuenta, cabe preguntarse cuál es la relación directa existente entre el programe de ingeniería de sistemas y la economía. Pues bien, es importante señalar que el programa formativo de ingeniería de sistemas estudia la creación de sistemas computacionales que sean óptimos y mantenibles, ofreciendo solución a las necesidades de los usuarios finales del sistema. Por tanto, al desarrollar un sistema de software es importante estudiar y evaluar aspectos como la eficiencia del software la aportación de valor para el cliente y además, el facil mantenimiento del sistema. En este sentido, al desarrollar un software hay que tener presente que se hace uso de recursos computacionales y eléctricos, así pues, entre más recursos consuma un sistema, más carga del procesador y demás hadwares necesita y por tanto más consumo eléctrico. De esta manera, se hace importante la aplicación de conceptos de economía en este proceso, debido a que se busca satisfacer las necesidades del cliente teniendo en cuenta los recursos limitados.

% Texto resumido
\normalsize Ahora bien teniendo esto en cuenta, la relación entre ingeniería de sistemas y economía radica en la eficiencia y valor agregado en el desarrollo de sistemas computacionales. En este sentido, el programa de ingeniería de sistemas se centra en crear soluciones óptimas y mantenibles para los usuarios finales. Por tanto, esto implica considerar la eficiencia del software, el valor para el cliente y el costo de mantenimiento. Así pues, dado que el desarrollo de software requiere recursos computacionales y eléctricos, se debe optimizar su consumo para minimizar la carga del procesador y el consumo energético. Por ende, la aplicación de conceptos económicos asegura la satisfacción del cliente dentro de los límites de recursos disponibles.
