\newpage
\clearpage
\justify
\fontsize{12pt}{14}
\setlength{\parindent}{0pt}

\section{Actividad 5}
\normalsize Dar respuesta a la siguiente pregunta tipo Saber Pro: En
economía se realizan diversas transacciones producto de intercambio de
bienes o servicios y se agrupan en diferentes nichos, pero
universalmente, tienen su desarrollo en la universalidad de los mercados.
¿De acuerdo lo anterior cual sería la definición de mercado?

\begin{enumerate}[label=\Alph*)]
    \item \colorbox{yellow}{\adjustbox{minipage=[t][\height][b]{0.8\linewidth}}{Es un conjunto de mecanismos mediante los cuales los compradores y vendedores de un bien o servicio están en contacto para comerciarlo.\strut}}
    \item Variables en los precios de compra.
    \item Leyes entorno a la producción.
    \item Actividades mercantiles especificas.    
\end{enumerate}

\normalsize La opción A es la correcta porque define al mercado como un entorno donde compradores y vendedores interactúan para comerciar bienes y servicios. Así pues, se refiere a los mecanismos que facilitan este intercambio, incluyendo factores como oferta, demanda, precios y regulaciones. Por tanto, esto captura la esencia del mercado como el lugar donde se realizan transacciones comerciales entre partes interesadas en comprar y vender productos o servicios. Mientras que, las opciones B, C y D no abordan adecuadamente esta definición integral del mercado en economía.